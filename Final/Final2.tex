\PassOptionsToPackage{unicode=true}{hyperref} % options for packages loaded elsewhere
\PassOptionsToPackage{hyphens}{url}
%
\documentclass[]{article}
\usepackage{lmodern}
\usepackage{amssymb,amsmath}
\usepackage{ifxetex,ifluatex}
\usepackage{fixltx2e} % provides \textsubscript
\ifnum 0\ifxetex 1\fi\ifluatex 1\fi=0 % if pdftex
  \usepackage[T1]{fontenc}
  \usepackage[utf8]{inputenc}
  \usepackage{textcomp} % provides euro and other symbols
\else % if luatex or xelatex
  \usepackage{unicode-math}
  \defaultfontfeatures{Ligatures=TeX,Scale=MatchLowercase}
\fi
% use upquote if available, for straight quotes in verbatim environments
\IfFileExists{upquote.sty}{\usepackage{upquote}}{}
% use microtype if available
\IfFileExists{microtype.sty}{%
\usepackage[]{microtype}
\UseMicrotypeSet[protrusion]{basicmath} % disable protrusion for tt fonts
}{}
\IfFileExists{parskip.sty}{%
\usepackage{parskip}
}{% else
\setlength{\parindent}{0pt}
\setlength{\parskip}{6pt plus 2pt minus 1pt}
}
\usepackage{hyperref}
\hypersetup{
            pdfborder={0 0 0},
            breaklinks=true}
\urlstyle{same}  % don't use monospace font for urls
\usepackage{graphicx,grffile}
\makeatletter
\def\maxwidth{\ifdim\Gin@nat@width>\linewidth\linewidth\else\Gin@nat@width\fi}
\def\maxheight{\ifdim\Gin@nat@height>\textheight\textheight\else\Gin@nat@height\fi}
\makeatother
% Scale images if necessary, so that they will not overflow the page
% margins by default, and it is still possible to overwrite the defaults
% using explicit options in \includegraphics[width, height, ...]{}
\setkeys{Gin}{width=\maxwidth,height=\maxheight,keepaspectratio}
\setlength{\emergencystretch}{3em}  % prevent overfull lines
\providecommand{\tightlist}{%
  \setlength{\itemsep}{0pt}\setlength{\parskip}{0pt}}
\setcounter{secnumdepth}{0}
% Redefines (sub)paragraphs to behave more like sections
\ifx\paragraph\undefined\else
\let\oldparagraph\paragraph
\renewcommand{\paragraph}[1]{\oldparagraph{#1}\mbox{}}
\fi
\ifx\subparagraph\undefined\else
\let\oldsubparagraph\subparagraph
\renewcommand{\subparagraph}[1]{\oldsubparagraph{#1}\mbox{}}
\fi

% set default figure placement to htbp
\makeatletter
\def\fps@figure{htbp}
\makeatother


\date{}

\usepackage[style=verbose-ibid,backend=bibtex]{biblatex}
\addbibresource{lib.bib}
\usepackage{ftnxtra}
\usepackage{fnpos}
\interfootnotelinepenalty=10000
\usepackage{fontspec}
\setmainfont{Roboto-Light.ttf}[
BoldFont = Roboto-Medium.ttf ,
ItalicFont = Roboto-LightItalic.ttf]
\begin{document}


\tableofcontents
%\listoffigures
\subsection{References}
\nocite{*} 
\printbibliography[heading=none]

\title{Religion without a Row: Facilitating Positive Religious Dialogue
Online}


\newpage 

\textbf{Background}

When approaching this topic, we need to have background into two areas;
the landscape of religious beliefs, and the landscape of the online
world.

We will start with religious belief. Worldwide, over 80\% of the world
is religious, with Christians and Muslims making up over 50\% of the
world's population\autocite{HackettChristiansremainworld2017}. However,
it is important to note that this does not mean 80\% of the world is
committed to their faith. The terms `culturally Muslim' or `culturally
Christian' have been used to describe these individuals. And they can be
surprisingly common. In a study on Islam by the Pew Research
centre\autocite{PewResearchCenterWorldMuslimsUnity2012}, only 18\% of
Kazakh Muslims said that their faith is very important in their lives
(the full results can be seen in Figure 1). This is significant, since
the Kazakh Muslim population is over 12 million people. The conclusion
made is that:

\begin{quote}
``Even though the idea of a single faith is widespread, the survey finds
that Muslims differ significantly in their assessments of the importance
of religion in their lives, as well as in their views about the forms of
worship that should be accepted as part of the Islamic
faith.''\autocite[pg 16]{PewResearchCenterWorldMuslimsUnity2012}
\end{quote}

So, when looking at religious groups, we must remember the shades of
grey in belief. One cannot cleanly divide people into religious and not
religious - it can be considered a spectrum of how important the
religion is to them.

\begin{figure}
\centering
\includegraphics{/Users/raviwoods 1/GoogleDrive/MainDrive/IDE/CHS/CHSDissertation/Final/CulturallyMuslim.png}
\caption{Percentage of Muslims from different countries that think their
religion is very
important\autocite[pg 8]{PewResearchCenterWorldMuslimsUnity2012}}
\end{figure}

But what about the doctrine of these major world religions? We will look
at two areas that are important to our study, through the lens of three
major world religions; Christianity, Islam and Hinduism. The first area
is the belief in an afterlife, which helps us understand how religious
people see those who are not of their faith.

In Christianity, there is a belief that all humans are evil and sinful,
but ``the sacrificial death of Jesus Christ reconciled humanity with God
and made salvation possible''\autocite[pg 59]{Boyett12MajorWorld2016}.
While the actual process of salvation is varied among Christian
traditions, ``most agree it begins with belief that Jesus is the son of
God and that he was resurrected from the
dead''\autocite[pg 59]{Boyett12MajorWorld2016}. Without belief in these
central tenets, then, salvation is not possible. Islam, in contrast,
does not have a belief that humans are inherently evil. They do believe
in salvation, however, where ``salvation is viewed as entrance to jannah
(heaven), given to those who believe in Allah and Islam, and whose good
deeds outweigh their sins''\autocite[pg 85]{Boyett12MajorWorld2016}.
While Christianity makes specific beliefs necessary for entrance into
heaven, Muslims disagree on whether (for example) a Christian who
beliefs in the Christian God can get into jannah. Finally, Hinduism is
different entirely, where there is not a specific heaven after death,
and beliefs do not play much factor. Instead, they focus on Karma:

\begin{quote}
Karma is the sum of a person's actions throughout samsara, or the cycle
of birth and rebirth. Good actions result in good karma and a favorable
rebirth. Negative actions result in bad karma and unfavorable
consequences in the next life. Though death may destroy a physical body,
the inner self, or atman, survives---only to be born again in another
form.\autocite[pg 104]{Boyett12MajorWorld2016}
\end{quote}

This knowledge, in some way, shows us how people of these religions will
interact with those of other faiths. Christians will, on the whole,
encourage non-Christians to learn about Jesus. It cannot necessarily be
said their goal is conversion, since they know God is the one who
converts. Instead, they usually want non-Christians to have an
open-minded examination of Jesus. This is of course different between
different Christians, cultural or otherwise. Muslims will be similar, in
part, however there is a stronger focus on charitable causes. This may
mean interacting with non-Muslims because of charitable causes, or
encouraging non-Muslims to take part in charitable causes. Finally,
Hindus will care much less about the beliefs of non-Hindus, instead
being committed to getting them to do good deeds.

The second area is the holy texts of a religion. Most religions have
some form of holy text. Most Christians believe in a divinely inspired
Bible, with some traditions adding extra divinely inspired material.
Catholics, for example, believe that the Pope's words are infallible and
divine. As for Muslims, they ``regard the Quran as the word of
Allah''\autocite[pg 83]{Boyett12MajorWorld2016}. In addition, there are
the hadith - ``Muhammad's sayings on a variety of
issues''\autocite[pg 83]{Boyett12MajorWorld2016}. As for Hindus, while
they do not believe in a specific holy text, ``several Sanskrit texts
are believed either to be divinely inspired or to have always
existed''\autocite[pg 104]{Boyett12MajorWorld2016}. In dialogue with
religious people, then, it is crucial to understand how fundamental
these texts are to committed believers. By appealing to these texts, we
can have an effect on the behaviours and motivations of individuals in
that religion. This will be important in Section X.

Hopefully, this brief overview of the landscape of 21st century
religious belief gives us the background knowledge for our research. But
what about the online world? How has the way we use the Internet
changed? In the early days of the internet, people began to see it as a
tool to bring people together; as a tool to solve issues of negative
dialogue. Way back in 1993, internet theorist Michael Hauben wrote that
``The Net brings the isolated individual into contact with people,
opinions, and views from the rest of the
world''\autocite{HaubenNetNetizensImpact1993}. This, he concludes, is an
important aspect of the online world, since ``exposure to many possible
opinions gives the reader a chance to actually think something over
before making a decision as to a personal
opinion''\autocite{HaubenNetNetizensImpact1993}.

However, as the internet grew, each user's journey through the online
world was tailored just for them. Each website began to create so called
filter bubbles for each user. These fundamentally altered the way we
encounter ideas and information, with content becoming personalised to
us. Machine learning algorithms, on many websites, learn what we enjoy,
and tailor content for us. While we might expect this in, for example, a
music streaming website, these algorithms begun to be used in diverse
areas. For example, in Parisier's book \emph{The Filter Bubble,} wrote
that as ``Google personalized for everyone, the query `stem cells' might
produce diametrically opposed results for scientists who support stem
cell research and activists who oppose
it''\autocite[pg 2]{PariserFilterBubblewhat2012}, and yet most people
don't even know. The same is true in news sites, social media, and
almost anywhere we consume content online. So, rather than being exposed
to different opinions like the early internet, echo chambers form;
communities with ``little variance in opinion\ldots{}where there is no
desire, or a means, to access a different point of
view''\autocite{Thwaitenewtheoryecho2018}. This is the current landscape
online; individuals becoming increasingly siloed based on their views,
often without them knowing. With this knowledge and background, then, we
can begin to go deeper into the central issues underlying online
religious dialogue.

But what is my personal motivation? For me, as a Christian, I love to
engage in dialogue about religion. I don't feel like I have been
brainwashed into my faith, and actually came to Christianity because of
the evidence for it. However, it can sometimes feel like religion is
never discussed between western millenials. In personal experience, this
is especially true online, where the only times I see it discussed is
around the `hot issues'; LGBTQ+ issues, abortion, euthanasia, etc. And I
think that the online landscape is a big factor in this. To me, the
internet has the opportunity to be a place where you can open dialogue
with someone in a totally different context, with totally different
views to yours. Yet, so often dialogue turns sour, towards argument and
anger. This, I think, is a great shame. While I don't think this piece
will solve the issue, I hope it helps me (and perhaps you, the reader),
to understand it more.

\newpage 

\textbf{Introduction}

Religion, whether you like it or not, is a huge influence on the world's
population. While the number of people calling themselves athiests has
increased, especially in the West, over the past half century (see
Figure 2), it is difficult to deny to importance of religion in the
public sphere. Over 80\% of the world is religious, with Christians and
Muslims making up over 50\% of the world's
population\autocite{HackettChristiansremainworld2017}. Whether
positively or negatively, it is difficult to deny that these beliefs
effect our society as a whole; our art, culture, entertainment, laws and
politics are all markedly changed by religious values. Even though
portions of the world's religious population are only culturally so (as
discussed in Section X), these religions still have an affect on these
individuals. And for those who are committed believers, the religion
does not just affect their actions within a church or a mosque - but
their day-to-day actions and thoughts also.

\begin{figure}
\centering
\includegraphics{./NonesRise.jpg}
\caption{Graph showing the rise of Americans with no religious
identification, from Gallup\autocite{GallupThisEasterSmaller}}
\end{figure}

And yet, our clearest picture of someone practising Islam, for example,
is often found in suicide bombers, rather than in the practices of those
close to us. The comedian Lee Mack makes this point, when being
interviewed on the BBC Radio 4 show \emph{Desert Island Discs}:

\begin{quote}
``I think it's quite odd that people like myself, in their forties, are
quite happy to dismiss the Bible, but I've never read it. I always think
that if an alien came down and you were the only person they met, and
they said, `What's life about? What's earth about? Tell us everything,'
and you said, `Well, there's a book here that purports to tell you
everything. Some people believe it to be true; some people {[}do{]} not
believe it {[}to be{]} true.' `Wow, what's it like?' and you go, `I
don't know, I've never read it.' It would be an odd thing wouldn't
it?''\autocite{BBCRadio4DesertIslandDiscs13}
\end{quote}

Mack points out the attitude of dismissing the Bible without examining
it is, at face value, an odd thing to do. While we may be scared off by
the actions of the religious, or we feel constrained by the taboo around
religion, not bothering to ever look into it leaves us in the dark; we
end up distancing ourselves from the people around us.

However, at times, we might think that discussing God or religion will
be fruitless; that it will only end in debate, shouting and, ultimately,
an impasse. Just look to any online comments section! (see Figure 3)

\begin{figure}
\centering
\includegraphics{./NewspaperComments.png}
\caption{A number of comments below an article on China's banning of the
Bible, found in 5 minutes from browsing the front page of \emph{The
Independant}.\autocite{OppenheimChinacrackssales2018}}
\end{figure}

But this is not a new issue. Herodotus, an Ancient Greek historian,
tells of a similar impasse when two groups discuss burial customs over
2000 years ago:

\begin{quote}
``When Darius was king, he summoned the Greeks who were with him and
asked them what price would persuade them to eat their fathers' dead
bodies. They answered that there was no price for which they would do
it. Then he summoned those Indians who are called Callatiae, who eat
their parents, and asked them (the Greeks being present and
understanding by interpretation what was said) what would make them
willing to burn their fathers at death. The Indians cried aloud, that he
should not speak of so horrid an
act''\autocite{HerodotusHistoryHerodotus1910}
\end{quote}

Karl Popper, in \emph{The Myth of the Framework}, rejects the notion
that this confrontation was fruitless. While he agrees that ``mutual
understanding was not
achieved''\autocite[pg 37]{PopperMythFrameworkdefence1997}, he points
out that even without conversation, this confrontation can begin to
breed tolerance and respect to those different from ourselves and, over
time, this can bear fruit; the fruit of
understanding\autocite{PopperMythFrameworkdefence1997}. This is what is
truly important - understanding. While we may never agree with someone
of with different beliefs to us, we may at least be able to empathise
with them if we understand their position.

And the internet can help with that. It is, at its core, a platform that
connects people. The internet allows people with totally different views
to interact and understand each other. Yet, as seen in Section X, the
modern internet is increasingly personalised, so the we are rarely
exposed to views we disagree with. When it comes to religion, then, how
can we push back against this over-personalisation? How can we get back
to an agora-like Internet, where the world can meaningfully discuss
issues of religion and philosophy?

\newpage 

\textbf{What stops us using Bohm Dialogue in online religious dialogue?}

Bohm Dialogue is a theory of dialogue put forth by David Bohm, in his
book \emph{On Dialogue}\autocite{BohmDialogue2004}. We will first
examine his theory. However, his theory was not written for the online
world. So, we must also examine the issues that we face when moving Bohm
dialogue online.

In \emph{On Dialogue}, the initial distinction made by Bohm is between
discussion and dialogue. Bohm points out that the word discussion has
``has the same root as `percussion' and
`concussion'.''\autocite[pg 7]{BohmDialogue2004}, and so emphasises
breaking things up and analyzing them, in order to come to one
consensus. This, he says, leads to a ping-pong game, where individuals
are aiming to score points for your
side\autocite[pg 7]{BohmDialogue2004}. Dialogue, on the other hand,
comes from two roots - ``\,`dia' which means `through' and `logos' which
means `the word', or more particularly, `the meaning of the
word.'\,''\autocite{BohmDialogueProposal}. This conjures the image of a
river of meaning, flowing through and around individuals engaged in
dialogue. Bohm proposes that this flow of shared meaning does two
things. Firstly, it creates a new understanding between participants,
where before (with discussion) there was a divide. Secondly, it focuses
on creating something new - a new flow of shared meaning.

To create this kind of dialogue, assumptions must be addressed. When
individuals come together, there will be a variety of held assumptions
and opinions underlying the conversation. When a particular assumption
from one member comes up, another member may be angry, for example with
the Greeks and the Callatiae seen in Section X. Bohm calls members in
dialogue to `suspend' an angry reaction (unkind words, for example).
Bohm says that

\begin{quote}
``{[}Suspension{]} involves exposing your reactions, impulses, feelings
and opinions in such a way that they can be seen and felt within your
own psyche and also be reflected back by others in the
group''\autocite{BohmDialogueProposal}
\end{quote}

On the one hand, this allows you to feel like you have sated the anger
in some way and, on the other, it allows the group to give serious
examination to why individual thoughts and assumptions give rise to
strong emotions and feelings\autocite{BohmDialogueProposal}. While Bohm
says that religious dialogue is often the most difficult, this notion of
suspension helps tremendously. In discussion or debate, individuals are
trying to convince the group of some kind of truth position (a religious
one, or otherwise). However, Bohm dialogue recognizes there will be
clashes and anger over certain issues and certain assertations. So,
through suspension, it seeks to help members explore where this anger
comes from.

This new model of dialogue, where focus is on shared meaning and
suspending reactions to others' opinions, is set out to work in the
general case. However, in our context, there is a barrier we need to
examine. The online world is markedly different to the offline world
Bohm was writing in, and these changes pose problems to dialogue, as
well as creating new opportunites for it. We will thus examine the
application of Bohm dialogue in the online world.

One major area is anonymity and masking. Some platforms are wholly
anonymous. In a forum or in a game, for example, users create a new
avatar for themselves that can be similar or different to their offline
self. In some ways, this anonymity allows for more open dialogue. In a
2001
study\autocite{JoinsonSelfdisclosurecomputermediatedcommunication2001},
students were asked to talk about personal questions in pairs; some
pairs spoke in person, while others spoke anonymously though an instant
messaging program. After this study, it was concluded that the online
students ``disclosed signifcantly more about
themselves''\autocite[pg 181]{JoinsonSelfdisclosurecomputermediatedcommunication2001}
than those who were face-to-face. In many ways this makes sense. There
are a number of social pressures dictating behaviour in offline
conversation, while those pressures are almost abstracted away when
conversation just becomes text on a screen. However this can have a
negative side also. When, for example, Bohm asks people to suspend their
angry reaction to someone's opinion, there is a social pressure to do
that, especially since rude words and shouting is mostly frowned up in
conversations. However, those pressures don't have the same weight in
the offline world, due to the abstraction of the interface. This is
likely why we often see arguments below news stories and YouTube videos.

In many ways, the situation is similar in non-anonymous platforms (such
as messaging friends, or using social media). While we may know the
person, there is still some sort of abstraction. Seeing someone we know
in person makes us tighten up to social pressures, while interacting
with them on the internet makes us freer on the whole. Another important
area to examine is why people use social media. An example is a 2012
study, which examined why people use
Facebook\autocite{NadkarniWhypeopleuse2012}. They found that Facebook
use is motivated both by the need to belong, and by the need for
self-presentation. The first of these gives rise to relationship and
community bonding. The second gives rise to holiday photos and bragging.
The combination of these, however, is not dialogue. Instead, individuals
become part of a community, and then are afraid discuss anything
challenging that community. So then, while the nature of the internet
does in part allow for freer conversation, relational social media sites
can give rise to tight communities, not dialogue.

Another issue is how information is displayed. In Bohm Dialogue, the
``basic notion\ldots{}would be for people to sit in a circle {[}since{]}
such a geometric arrangement doesn't favor
anybody''\autocite[pg 17]{BohmDialogue2004}. In social networks this is
often not the case. The `sitting' in the online space is akin to what we
see when we login to the platform. Most platforms show us, at least
automatically, content based on one of two factors. The first is what
most interests us -- found on YouTube, and the majority of the screen on
Twitter and Facebook. This is usually based on machine learning
algorithms which learn which content we are most likely to click, share
or like. The second is what is most popular among the whole of the
network (in other words, what is trending) -- found on Reddit, as well
as a smaller section of the screen on Twitter and Facebook. But this
arrangement does favour certain people. In the first, it favours people
with views that are more popular to the most number of people. In the
second, it favours those that are popular among the whole network. This
leads to, on the whole, populists being favoured in internet
discussions. I would suggest that a more democratic, more `circle-like'
configuration may be to prioritise by the newest content. The issue here
is that this approach will likely cause users to spend less time on the
site when compared with the other approaches, since the latter is
designed, through software, to keep us on the site longer. This is
evidenced by the fact that many social networks are beginning to sort
their feeds by this latter approach. Twitter, for example, changed the
makeup of its timeline in February 2016, with it now ``designed to
{[}show{]} the best tweets that users may have missed based on what
Twitter thinks you care about''\autocite{LynleyTwitterWillNow2016},
rather than showing the most recent tweets.

These two areas make it harder to move Bohm dialogue online. The
abstraction of the internet makes us more likely to sate our anger
through rude words, and the current structure of online content is
populist, not democratic. However, it is not impossible, and we can
still meaningfully utilise the principles of Bohm dialogue in the rest
of this piece.

\newpage 

\textbf{How does the Bible encourage religious dialogue?}

There is a tendency I have noticed, both online and offline, for
Christians to bubble off into their own cliques and communities. In this
piece, I seek to argue that this tendency is not one that comes from the
Bible and, in fact, Christians should embrace an agora-like Internet as
a chance to share their faith. To achieve this, it makes sense to use
the text that unites the Christian population - the gospels. These four
books are four accounts of the life of the central figure of
Christianity, so turning to these seems sensible. Of course, there will
be some Christians who see the gospels as a guide, rather than as holy
books. This section will likely not have the same bite for them. Here,
for simplicity, we will use just two short sections from the gospel of
John. While there is little context in these texts, both support one
another in what they say. The first is chapter 3 verse 16, likely the
most famous verse from John's gospel, often seen around stadiums during
American sports games. The verse itself is a clear and concise
description of Christ's role in the faith:

\begin{quote}
``For God so loved the world that he gave his one and only Son, that
whoever believes in him shall not perish but have eternal
life.''\autocite[pg 1035]{HolyBibleNew2007}
\end{quote}

\begin{figure}
\centering
\includegraphics{./TimTebow.jpg}
\caption{American footballer Tim Tebow, with John 3:16 painted below his
eye during a playoff game}
\end{figure}

The second is chapter 30, verses 30 and 31. This comes near the end of
the gospel, and is an explanation by John as to why he curated the signs
(that is, miracles) of Jesus the way he did:

\begin{quote}
``Jesus performed many other signs in the presence of his disciples,
which are not recorded in this book. But these are written that you may
believe that Jesus is the Messiah, the Son of God, and that by believing
you may have life in his name.''\autocite[pg 1057]{HolyBibleNew2007}
\end{quote}

From these two verses we see three beliefs central to the Christian
faith:

\begin{enumerate}
\def\labelenumi{\arabic{enumi}.}
\tightlist
\item
  God has one son, Jesus, who he gave to the world.
\item
  This son, Jesus, is messianic. That is to say, he is some sort of
  saviour figure in Christianity.
\item
  If you believe in Jesus as the Messiah, and as God's Son, you can have
  eternal life in his name.
\end{enumerate}

The important belief to us here is the third. It is clear that belief in
Jesus is important to Christians; to believe in Jesus is to gain access
to an everlasting life after this one. Then, for the Christian, the role
of dialogue is to help others to know and understand Jesus. Some may
call this dialogue proselytizing. However, proselytizing brings up
images of megaphones on street corners; proselytizing is coercive and
pushy. Dialogue, on the other hand, is the Christian sharing their
faith, answering questions and so forth, in order to help people make up
their mind about Jesus properly. So, when Jesus says to ``love your
neighbor as yourself''\autocite[pg 956]{HolyBibleNew2007} in Matthew's
Gospel, I would call this form of dialogue more loving than the man
shouting on the street corner.

\begin{figure}
\centering
\includegraphics{./StreetPreaching.jpg}
\caption{A street preacher with a megaphone.}
\end{figure}

However, I would argue, to not share your faith as a Christian is also
an unloving act. In doing this, the Christian believes that they have
eternal life, yet they don't want anyone around them to have that life
also. Penn Jillette, Las Vegas magician and advocate for atheism, agrees
with this sentiment. In a video on the subject he said this:

\begin{quote}
``If you believe there is a heaven and hell, and people could be going
to hell or not getting eternal life or whatever, and you think it's not
really worth telling them this because it would make it socially
awkward\ldots{}How much do you have to hate somebody to not {[}tell
them{]}?'' \autocite{JilletteGiftBible2010}
\end{quote}

When we look online though, we see that it is easy for anyone, Christian
included, to stay in a bubble online. Eli Parisier, in \emph{The Filter
Bubble}, says that these exist because of the personalisation algorithms
found across the web. However, Parisier says, the bubble is ``a cozy
place, populated by our favorite people and things and
ideas''\autocite[pg 12]{PariserFilterBubblewhat2012}. Ultimately, like
anyone, Christians can be scared of what people will think of them, and
they don't like being challenged. In addition, with religion
specifically, this issue isn't solely who you engage with on Facebook;
almost all your media consumption can be within a Christian bubble. At a
conference, Mark Scott, the Former Managing Director of the Australian
Broadcasting Corporation, explained the issue as follows:

\begin{quote}
``The new media environment presents a great risk for Christians to
retreat. There will be in a media sense, a massive global market for
Christians to listen to Christian music, to read Christian books, to see
Christian films, to partake in Christian blogs, to comment on each
other's Christian Facebook pages and to live in that Christian
world.''\autocite{TaylorMarkScottChristians2014}
\end{quote}

For Christians then, there is tremendous comfort in staying within the
bubble, and there is enough media to allow you to stay there for as long
as you want to. Thus there seems to be a clash in the minds of
Christians, between (somewhat selfishly) staying within the bubble, and
(more selflessly) sharing your faith for the sake of those around you.

This clash can be seen in a 2017 study by Brubaker and
Haigh\autocite{BrubakerReligiousFacebookExperience2017}. In the study,
335 Christians participated in an online study about their engagement in
religious content and community online. With regards to how much
Christians see Facebook as a platform for dialogue, they found that
``those who use {[}Facebook{]} for religious purposes recognize the
potential for visibility and therefore reach out to people with diverse
beliefs and varying commitments to those
beliefs''\autocite[pg 8]{BrubakerReligiousFacebookExperience2017}.
However a second, more interesting insight is that ``people who were
more religious were also more likely to minister to others
online''\autocite[pg 9]{BrubakerReligiousFacebookExperience2017}. This
seems to back our hypothesis above; those who are more religious are
more certain of an everlasting life after this one, so will think it
more crucial to try to tell people about Jesus, and that new life. In
contrast, those who are less sure themselves, are more likely attracted
to the comfort the bubble provides, rather than sticking their neck out
for the sake of those around them.

So, from this, we have seen that the Bible backs up the case for
dialogue (rather than proselytizing), and yet some Christians can feel
conflicted. On the one hand, they want to start a dialogue out of a
sense of love for those around them. Yet, there is comfort in staying
still, and dangers (either percieved or real) of sharing their faith
online. It is my opinion, then, that we need to motivate these Christian
with what the Bible says about dialogue. As online personalisation seems
to be only getting stronger, Christians will have a tendency to clump
together, unless we can help them understand why that tendency is an
unbiblical one.

\newpage 

\textbf{How does network theory help disparate religious groups to
interact?}

'Birds of a feather flock together" as the saying goes. This is
homophily; the tendency of individuals to associate with those similar
to them. While homophily is hardly a new concept, the dawn of social
networks provided an extensive dataset to study the area. In 2008,
Thelwall looked at a sample of 2,567 members of Myspace to see patterns
of behaviour\autocite{ThelwallHomophilyMySpace2009}. While he found no
evidence of homophily within genders, he found significant evidence of
homophily in many other areas, including within
religions\autocite[pg 229]{ThelwallHomophilyMySpace2009}. However,
social networks do more than just provide data; they change the very
nature of the connection between members within the network. In this
section, I seek to show that individual tendency, coupled with the
structure of social networks (looking at Facebook specifically), only
seeks to clump people together. In addition, I propose two possible
areas that could successfully push back against this model.

In \emph{The Filter Bubble}, Parisier says that online filter bubbles
``tend to dramatically amplify confirmation
bias''\autocite[pg 88]{PariserFilterBubblewhat2012}. Since we naturally
become frustrated by information that challenges our assumptions, we
tend to instead drift towards information that we agree with. Thus, we
have a tendency toward those who hold a similar viewpoint to us; those
of the same religion, or even of the same denomination within that
religion. And, since online filter bubbles personalise, they amplify
things we have a tendency towards, so amplifying this confirmation
bias\autocite[pg 88]{PariserFilterBubblewhat2012}.

But how does this compare to the offline world of homophily? Take the
example of stratified housing communities, where the rich and the poor
live in different districts. Each district is like it's own filter
bubble, amplifying confirmation bias within it. However, the difference
lies in the fact that each member is not confined to their own district.
Naturally, people live in different contexts, and move between these
contexts daily. While these contexts may be related (those who are rich
may have different hobbies to those who are poor), each context is
skewed in different ways (as can be seen in Figure 6). So, while your
affinity toward certain people still exists (as seen by the thickness of
the lines in the figure), you end up interacting with people from
different religious beliefs.

\begin{figure}
\centering
\includegraphics{./NetworkDiagram1_1.png}
\caption{Offline contexts of a Christian, with affinity toward a person
indicated by line thickness}
\end{figure}

In the offline world, however, you hold one identity - one profile.
Facebook founder, Mark Zuckerberg told journalist David Kirkpatrick for
his book \emph{The Facebook Effect:}

\begin{quote}
``The days of you having a different image for your work friends or
coworkers and for the other people you know are probably coming to an
end pretty
quickly.''\autocite[pg 119]{KirkpatrickFacebookeffectstory2011}
\end{quote}

So it makes sense that, on Facebook, there is one context where you have
no direct control. And, when all the contexts are aggregated (see Figure
7), online filter bubbles amplify those you have an affinity for. We see
then that, online, your feed becomes skewed towards views similar to you
in a different way to the offline world.

\begin{figure}
\centering
\includegraphics{./NetworkDiagram1_2.png}
\caption{The aggregation of offline contexts onto a single online
profile. Now, affinity towards other Christians shows up much more
starkly, since these are the only people you see in your feed.}
\end{figure}

However, within this network, it might seem surprising that the number
of hops between you and every other member is actually very small. In a
2011 analysis of the Facebook network, they found that 99.6\% of users
are connected in 6 links or less, with the average distance being 4.7
links\autocite[pg 4-5]{UganderAnatomyFacebookSocial2011}, as seen in the
graph in Figure 8. At the same time, though, they found that the amount
of clustering in Facebook is very high. In the literature, clustering is
measured as a coefficient between 0 and 1. A coefficient of 1 indicates
that all of your friends are also friends with each other. In the 2011
analysis, they concluded that ``for users with 100 friends, the average
local clustering coefficient is 0.14, indicating that for a median user,
14\% of all their friend pairs are themselves
friends''\autocite[pg 6]{UganderAnatomyFacebookSocial2011}. This
coefficient as found to be ``five times greater than the clustering
coefficient found in a 2008 study analyzing the graph of MSN messenger
correspondences, for the same neighborhood
size''\autocite[pg 6]{UganderAnatomyFacebookSocial2011}

\begin{figure}
\centering
\includegraphics{FBLinks.png}
\caption{Graph showing the percentage of user pairs that are within
\emph{h} hops of each other, from Ugander et
al.\autocite[pg 4, fig 2]{UganderAnatomyFacebookSocial2011}}
\end{figure}

This apparent contradiction is explained by a seminal paper by Strogatz
and Watts\autocite{WattsCollectivedynamicssmallworld1998}. Here they
called these networks, with a high amount of clustering and a small
average path length, `small-worlds'
networks\autocite[pg 440]{WattsCollectivedynamicssmallworld1998}. These
networks are ``caused by the introduction of a few long-range
edges''\autocite[pg 4]{WattsCollectivedynamicssmallworld1998};
individuals who have a supremely large number of links within the
network. These individuals, who we will call `hubs', become the `glue'
between disparate clusters. So, then, one approach to stop
over-personalisation might be to harness the power of these hubs. While
clustering is very high within the network, the 2011 study found an
interesting insight when studying friends-of-friends. While you would
expect the average user with 100 friends to have ``100∗99 = 9,900
non-unique
friends-of-friends''\autocite[pg 8]{UganderAnatomyFacebookSocial2011},
they found that they have far more than that; ``27,500 unique
friends-of-friends and 40,300 non-unique
friends-of-friends''\autocite[pg 8]{UganderAnatomyFacebookSocial2011}.
This is likely due to the these hubs in the network. While most of your
friends will have a similar number of friends as you, a small number are
incredibly well-connected, which explains why you have so many more
friends-of-friends than expected.

However, rather than playing this up, social networks tend to play this
down. Parisier, in \emph{The Filter Bubble}, looked at Twitter. He found
that:

\begin{quote}
``Twitter users see most of the tweets of the folks they follow, but if
my friend is having an exchange with someone I don't follow, it doesn't
show up. The intent is entirely innocuous: Twitter is trying not to
inundate me with conversations I'm not interested in. But the result is
that conversations between my friends (who will tend to be like me) are
overrepresented, while conversations that could introduce me to new
ideas are obscured.''\autocite[pg 150]{PariserFilterBubblewhat2012}
\end{quote}

So, then, one method would be to push against this shift within our
social networks, by allowing users to see interactions between their
friends and people they don't know, to give a springboard for
conversation with those outside your immediate cluster.

Coming back to contexts, a second method could be to split the internet
back into different contexts, while Facebook (among others) tend to
group them together. An example is the forum site Reddit, where there
are a number of smaller forums (called subreddits). The front page of
the subreddits you join are aggregated, to form your feed. The
difference with this compared to Facebook, for example, is that your
feed is not altered based on which subreddits you look at regularly, so
your experience is much more broad.

The issue here is one of framing. Imagine a user joins the subreddit for
badminton players, and the subreddit for Christians. This may not seem
like a problem, because these two communities are disparate; badminton
players are religiously diverse, and Christians play a lot of different
sports. However, subreddits are framed in a certain way; the badminton
subreddit is devoted to badminton, and the Christianity subreddit is
devoted to Christianity. Almost by definition, discussion on the
badminton subreddit is devoted to badminton, while discussion on
Facebook with your badminton friends, on the other hand, can be diverse.
This is likely because you are more comfortable around those friends you
play badminton with in the offline world; you know them personally, and
so want to find out about their life as a whole. However, the same
cannot be said about faceless users of a badminton forum.

The conclusion then might be to create new contexts online. A good
example is the DebateReligion subreddit\autocite{DebateReligion}, where
users (you guessed it) discuss and debate religious topics, or
ChangeMyView\autocite{ChangeMyView}, a more general subreddit for
discussing perspectives on different opinions. These subreddits are
designed as agora-like forums, to discuss and debate ideas. It is a rosy
picture, but the issue here is one of size. While ChangeMyView and
DebateReligion have roughly 600,000 combined subscribers, there are over
2 billion active Facebook
users\autocite{StatistaFacebookUsersWorldwide}. The user base of these
subreddits make up just 0.03\% of the user base of Facebook. However,
the model of specific communities of strangers dedicated to
understanding one another is a useful one.

Our second method is a happy medium between the ghetto-like model of
Facebook and the agora-like model of Reddit. Comparing the internet to a
city, Parisier says that ``we need our online urban planners to strike a
balance between relevance and serendipity, between the comfort of seeing
friends and the exhilaration of meeting strangers, between cozy niches
and wide open spaces''\autocite[pg 222]{PariserFilterBubblewhat2012}.

The network structure of Facebook (and other social networks) is flawed;
it clumps all those you know together, and then amplifies the connection
you feel towards those who hold the same beliefs you do. It creates, as
Parisier says, a ``city of
ghettos''\autocite[pg 222]{PariserFilterBubblewhat2012}. However, the
proposed two methods - context splitting, or harnessing the power of
hubs in the network - can help us to create a better online city, where
internet citizens are given a space to think and discuss the most
important questions of human existence.

\newpage 

\textbf{Conclusion: Moving the research towards design}

Having gone into these three research areas, we must then join the dots;
how can these areas come together into practical changes that facilitate
religious dialogue online? The first natural step is to look at the
design of the social networks themselves. The central issue is the echo
chambers that tend to form online, often as a result of the
personalizing of the content we see. As seen in Section X, an echo
chamber a community with ``little variance in opinion\ldots{}where there
is no desire, or a means, to access a different point of
view''\autocite{Thwaitenewtheoryecho2018}. In our specific context, this
central issue manifests itself in three ways.

The first is the amplifying of homophily, seen in Section X. In the
offline world, we tend to interact with those we identify with, and we
do this as religious groups also. However, we are still forced to
interact with those we don't identify with in the offline world; we have
little control over who sits next to us on the Tube, or who our
neighbours are, or who we work with, for example. In the online world,
however, we only see those one link away from us; those we are friends
with. We rarely see the wider network, and modern social networks are
only making this more rare, by reducing the interactions we see between
friends and friends-of-friends.

The second is what information is displayed, seen in Section X. On the
internet, we see one of two things; either the content that is trending
on the whole network, or content chosen for us, that we are likely to
enjoy. This is an incredibly populist approach, with little emphasis
given to minority and dissenting. For religious discussion, this is an
issue; we see the views of one or two major religions, without ever
seeing the views of the rest.

The final manifestion is in the sense of anonymity the internet
provides, also seen in Section X. Even on websites where we interact
with friends, the abstraction of the internet gives us a sense of
freedom to say what we feel. In part, this is helpful to let people
freely talk about their issues. However, it does not discourage
individuals from negative behaviours; argument, hate and anger.

In all these three areas, design decisions could be made to help. We
could see more of the wider network, or we could see more dissenting
views, or we could be dissuaded from negative behaviours in discussion.
Usually, however, the social networks themselves are acting to maximise
profit. The way they do this is not by encouraging positive dialogue; it
is by giving people what they want. By giving into an indiviudal's
cravings, the individual will spend more time on the site, and the
social network will get more advertising revenue. So, then, the we must
dig deeper, and look at the cravings and desires of people while they
are online. In general, what are people drawn to? Unfortunately, It
seems that there is still a disconnect between motivations that aid
healthy dialogue, and the things people are actually driven by. In a
paper by Berger and Milkman\autocite{BergerWhatMakesOnline2012}, they
looked at how viral different \emph{New York Times} articles. The
results can be seen in Figure 9. In general, it can be seen that when
high arousal emotions are evoked, content is popular. This can be
positive high arousal (awe and interest, for example). However, the most
popular content was not positive; it evoked anger. This is of course an
issue. The mindset for dialogue (as seen in Section X), is one of
openness; one of suspending angry emotions. This might go some way to
explain why discourse about religion online is often not like this; it
reverts to anger and argument quickly, simply because that's what piques
people's interest.

\begin{figure}
\centering
\includegraphics{/Users/raviwoods 1/GoogleDrive/MainDrive/IDE/CHS/CHSDissertation/Final/Screen Shot 2018-06-27 at 17.50.17.png}
\caption{Likelihood of virality for different elicited emotions in news
articles\autocite[pg 8, fig 2]{BergerWhatMakesOnline2012}}
\end{figure}

So, then, we must move on to ways of changing motivation in internet
users. We discussed motivation in part in Section X, where we could help
motivate Christians by going back to biblical doctrine. This could have
been done with those of other religions also. However, there are also
ways of changing motivation that don't appeal to holy books. In the
field of economics, often incentives are used to guide behaviour.
However these incentives don't have to be large. In behavioural
economics, a `nudge' is a small suggestion or reinforcement which guides
behaviour. Thaler and Sunstein, in their 2008
book\autocite{ThalerNudgeimprovingdecisions2008}, describe it as
follows:

\begin{quote}
``A nudge\ldots{}is any aspect of the~choice architecture~that alters
people's behavior in a predictable way without forbidding any options or
significantly changing their economic incentives. To count as a mere
nudge, the intervention must be easy and cheap to avoid. Nudges are not
mandates.''\autocite[pg 6]{ThalerNudgeimprovingdecisions2008}
\end{quote}

An example is given in the book of a fake image of a fly put into
urinals at Schipol airport, since ``if a man sees a fly, he aims at
it''\autocite{ThalerNudgeimprovingdecisions2008}. Thus, this small nudge
``reduced spillage by 80
percent''\autocite{ThalerNudgeimprovingdecisions2008}. By having these
small postive reinforcements and things to aim for, Thaler and Sunstein
argue, behaviour can be changed. Further work, then, needs to be done
into these nudges. Through experimentation and user-centred design, we
need to see how we can motivate someone towards open religious dialogue.
With Bohm dialogue as the target, how can we nudge online behaviour in a
way that makes people more willing to suspend their anger, in order to
engage in positive discourse? This is the research question that design
can answer.

\end{document}
